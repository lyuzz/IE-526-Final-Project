% Begin
\documentclass[12pt]{article}

% Packages
\usepackage[top=2cm, bottom=3cm, left = 1in, right = 1in]{geometry}
\usepackage[fleqn]{amsmath}
\usepackage{amssymb}
\usepackage{anysize}
\usepackage{graphicx}
\usepackage{subcaption}
\usepackage{float}
\usepackage{comment}
\usepackage{multirow}
\usepackage{xcolor}
\usepackage[
	backend=biber,
	maxcitenames=2,
	maxbibnames=10,
	citestyle=authoryear,
	bibstyle=authoryear,
	isbn=false,
	url=false,
	doi=false]{biblatex}
\renewbibmacro{in:}{}
\captionsetup[subfigure]{labelformat=empty}

% Bib file
\addbibresource{bibfile.bib}

% Paper style
\geometry{a4paper}

% Equation (section.equation)
\numberwithin{equation}{section}

% Title
\title{Report}
\author{Zefeng Lyu, Zeyu Liu, Rui Zhou}
\date{}

% Begin Document
\begin{document}

%MakeTitle
\maketitle

%Introduction
\section{Introduction}

	\paragraph{}Nowadays, electronic commerce offer us a new way for shopping and play an increasingly important role in people’s daily life. Many electronic commerce companies, such as Alibaba, Amazon have become famous all around the world. In order to satisfy customers and reduce the transportation cost, building an efficient and economic logistic system is significantly crucial. 
	
	\paragraph{}The Urban truck scheduling and routing problem talked in this article is a classic vehicle routing problem (VRP), which seeks to find routs to deliver goods from a central depot to a set of geographically dispersed customers \parencite{Coelho2016}. There exists a wide variety of VRPs and a broad literature on this class of problems \parencite{Laporte1992}. We could get a new VRP if only we change the side conditions, such as capacity restrictions, time windows, etc. It is notable that electric vehicles become more and more popular because of their property of environment-friendly and economic \parencite{Hiermann2016}. When electric vehicles are involved, problem will become more complicated because we need to determine whether a vehicle should get charged and move further or just turn back. When electronic vehicle is engaged, determining the time and station for recharging is crucial \parencite{Zhang2018}. Besides, some researchers pay more attention to our environment protection. Bektas and Laporte (2011) are the first to propose the Pollution-Routing Problem (PRP) to offer insights into environment-friendly vehicle routing. \parencite{Lin2014} upgraded the traditional objective to minimizing system-wide costs related to economic and environmental issues.
	
	\paragraph{}In this article, we study a real VRP derived from a Chinese electronic commerce company, named Jing Dong. The research focus on one of its distribution centers located in Beijing. This center serves around 1000 customers nearby. Finding a optimal routes is crucial for Jing Dong to reduce their transportation cost. The objective function are the sum of fixed cost, variable cost(transportation) and waiting cost. The constraints are listed as follows. 
	
	\begin{itemize}
		\item Capacity: Each vehicle has an identical volume and weight limitation.\item Time Window: Vehicles are supposed to arrive in the time windows. But early arrival is allowed at the cost of waiting.\item Distance range: limited by the energy, each truck has a maximum traveling distance. 
	\end{itemize}

	\paragraph{}We solve this problem by using heuristic algorithm and simulation. Heuristic algorithms for the VRP can often be derived from  the TSP(Laporte, 1992). There are lots of heuristic algorithms, such as Clarke and Wright algorithm, Tabu search algorithm, Genetic algorithm, Particle swarm algorithm, etc. However, when applying these methods to VRPs, care must be taken to ensure that only feasible vehicle routs are created \parencite{Laporte1992}. Simulation software is another helpful tool for VRP problem. AnyLogic, one of the most popular simulation software, can be used in to strengthen the model by introducing random traffic flow to simulate practical traffic conditions \parencite{Fan2009}. Besides, AnyLogic could visualize the result and make it readable for non-technical personnel.
	
	\paragraph{}The rest of this article is organized as follows. Section \ref{problem} defined the problem to be solved and put up the relative sample data. Section \ref{model} propose a mathematical model. Then, we describe the two algorithms we used in section \ref{algo}. After that, we feed the routes to “AnyLogic” and display it in section \ref{simu}. Finally, we list the result and give out some future research in section \ref{conclution}.  

\section{Problem definition}\label{problem}
	\paragraph{}A distribution center of Jingdong(JD) is associated to city Beijing, China. This center provides distribution services to over a thousand customers of this city needing bulk commodities daily. It is assumed that there is no constraint on how many trucks can be used. It is also assumed that JD’s comprehensive costs cannot be reduced. Each customer have a time window for receiving goods but it is a soft constraint, which means that track may arrival before than the time window. If vehicle arrives early, a waiting cost of 24 yuan per hour should be counted. Later arrival is not allowable. Each truck can serve multiply customers but each customer could only be served by one truck at one time. Besides, the capacity of truck should be taken into consideration when it serve multiply customers. The task is determining the quantity of truck to be used and designing their routing and scheduling plans. The final target is to optimize the total cost, which include fixed cost of using a truck, transportation cost and waiting cost if the truck arrive before the earliest time of that customer. 
	
		\subsection{Input data}
		% Table generated by Excel2LaTeX from sheet 'Sheet1'
		\begin{table}[htbp]
		\caption{Customer sample data}
 		\centering
    		\begin{tabular}{clrrrc}
    		
    	\hline
    Cust.ID    & Longitude & Latitude & Pack.Weight & Pack.Volume & LatestTime \\
    	\hline
    185   & 116.2030 & 40.09530 & 0.584525 & 1.0338 & 720 \\
    888   & 116.2381 & 40.10992 & 0.299000 & 2.3387 & 720 \\
    423   & 116.4810 & 39.79903 & 0.019570 & 0.0268 & 720 \\
    360   & 116.4042 & 39.87164 & 0.099740 & 0.1793 & 720 \\
    660   & 116.4783 & 40.01122 & 0.021600 & 0.0356 & 720 \\
    662   & 116.5157 & 40.02426 & 0.159990 & 0.3031 & 720 \\
    137   & 116.1088 & 39.75993 & 0.189045 & 0.3066 & 720 \\
    322   & 116.1369 & 39.70810 & 0.028260 & 0.0997 & 720 \\
    631   & 116.3299 & 40.07124 & 0.013400 & 0.0697 & 720 \\
    	\hline
    	
    		\end{tabular}
  			\label{tab:cust}
		\end{table}

		The distribution center was located at (116.571614, 39.792844). This distribution center is responsible for 1000 customers nearby. However, computing 1000 customer is complicated. In order to reduce the computing time, we simplify our model by randomly picking up 50 customers for analyzing. The data needed include the customers' longitude, latitude, package weight, package volume, and time window for receiving the cargo. 10 of these 50 data was displayed in Table \ref{tab:cust}. 

		% Table generated by Excel2LaTeX from sheet 'Sheet4'
		\begin{table}[htbp]
  		\centering
  		\caption{Vehicle capacity}
    		\begin{tabular}{cccccc}
    		\hline
  		  	Type  & Max\_Volume & Max\_Weight & Driving \_Range & Trans.\_Cost & Fixed\_Cost \\
  		  	\hline
 		   	TRUCK & 16    & 2.5   & 200   & 14    & 300 \\
 		   	IVECO & 12    & 2     & 120   & 12    & 200 \\
			\hline
    		\end{tabular}%
  		\label{tab:truck}%
		\end{table}%
		
		Each truck will leave from the distribution center after 8:00 AM and must return before 12:00 PM. There are two types of vehicle available. Type 1, named “IVECO”, could carry 2 tons of cargo while its volume is no more than 12 cubic meters. It can driving up to 100 kilometers. Its transport cost and fixed cost are 12 yuan per kilometer and 200 yuan per day. Type 2 named “TRUCK”. TRUCK is more powerful than "IVECO", but also more expensive. Its capacity of volume, weight and driving range are 16 cubic meters, 2.5 tons and 120 kilometers, respectively. Its transport cost and fixed cost are 14 yuan per kilometer and 300 yuan per day. Both of these two kind of vehicle are chargeable. But we do not take charging into consideration in this article. It is notable that we only use the type 2 vehicle “TRUCK” and its quantity is unlimited. The capacity information is shown in Table \ref{tab:truck}.
		
		% Table generated by Excel2LaTeX from sheet 'Sheet1'
		\begin{table}[htbp]
  		\centering
  		\caption{Distance sample between customers}
    		\begin{tabular}{cccc}
    		\hline
    		From  & To    & Distance & Time \\
    		\hline
    		185   & 888   & 8821  & 11 \\
    		185   & 423   & 60856 & 74 \\
    		185   & 360   & 46235 & 56 \\
    		185   & 660   & 37010 & 45 \\
    		185   & 662   & 38738 & 48 \\
   		 	185   & 137   & 49591 & 60 \\
    		185   & 322   & 52930 & 64 \\
    		185   & 631   & 17685 & 22 \\
    		185   & 689   & 59133 & 71 \\
    		185   & 559   & 48297 & 58 \\
    		185   & 550   & 52265 & 63 \\
    		\hline
    		\end{tabular}%
  		\label{tab:distance}%
		\end{table}%
		
		In order to calculate the transportation cost, we need the distance between distribution center and customer, customer and customer. Besides, we need the transport time to determine if the arrival time is in the time window. Part of the sample data are shown in Table \ref{tab:distance}. 
		
		\subsection{Constraints}
1) All customers should be served exactly once by one truck.\\
2) Capacity limitation: the whole weight and volume could not exceed the capacity of trucks.\\	
3) Time window: package should be delivered within the specified time window.\\
4) Each truck should return before sustainable mileage reaches 0. \\
5) Demand of customers must be satisfied.\\

		\subsection{Assumptions}	
1) The trucks begin at the distribution center and need to return to the distribution center once the goods have been delivered. \\
2) There is an infinite number of trucks available. \\
3) The unloading time for each customer is constant at 0.5h.\\
4) The coefficient of waiting cost described is 24 yuan/h. Later arrival is not allowable.
	
%Model
\section{Model}\label{model}
	
	\paragraph{}We model this problem as a Vehicle Routing Problem with Time Windows and Capacity Constraints (VRPTWCC)\parencite{Schneider2016}.  Consider a set of customers and a distribution center $\mathcal {N}=\{0,1,2,..., N\}$, in which 0 represents the distribution center and $1,2,...N$ represent customer locations. Deliveries happen on a graph $G(V, E)$, where $V=\{i: i\in \mathcal{N}\}$ denotes customer locations and the distribution center, $E=\{(i,j):i,j\in \mathcal{N}\}$ denotes the road between location $i$ and $j$.  Each customer $i$ has a package of volume $v_i$ and weight $w_i$. A truck is chosen from a set of trucks $\mathcal{K}=\{1,2,...K\}$ to load packages of several customers and deliver with the route that minimizes the cost. Here we assume that the number of trucks is given in advance. This assumption is relaxed later in simulation optimization. There is a fixed cost $c_f$ associated with each truck out for delivery and a travel cost $c_t$ accumulates per hour for each truck. Every hour that a customer waits for the delivery, a waiting cost $c_w$ will be generated to model the impatience of that customer. Trucks are assumed to travel in a constant speed, thus travel time between location $i$ and $j$ is also constant and already given. We use $d_{ij}$ and $t_{ij}$ to denote distance and travel time between location $i$ and $j$. When trucks are loading packages, the total volume and weight of packages should not exceed the maximal volume and weight, $V_{max}$ and $W_{Max}$, respectively. Powered by electricity, trucks have a travel limit $D_{max}$ before recharging. We assume that delivery starts at 9:00 am and customers expect to receive packages before 9:00 pm, which forms a delivery window of $T_{max}=12$ hours.
	
	\paragraph{}Route of a truck is chosen by a binary variable $x_{ij}^k$. If truck $k$ chooses to travel from $i$ to $j$, $x_{ij}^k$ is 1; otherwise it is 0. For the time windows, another variable $y_i$ is defined to record the time that customer $i$ is served. Note that in the model we choose the start time at 0, so that $y_0$ will always be 0. Objective is to minimize the total cost of operation and of customer waiting, which consists of three parts: fixed cost for each trucks, travel cost of each truck measured by time units and customer waiting cost also measured by time units. The problem can be formulated as a Mixed Integer Program.
	\begin{flalign}
	\text{\emph{Minimize}} \quad & \sum_{k\in \mathcal{K}}\sum_{j\in \mathcal{N}}c_f x^k_{0j}
			+\sum_{k\in\mathcal{K}}\sum_{i\in \mathcal{N}}\sum_{j\in \mathcal{N}\backslash\{i\}}c_d d_{ij}x^k_{ij} + \sum_{i\in \mathcal{N}\backslash\{0\}}c_w y_i ,\\
	s.t. \quad  & \sum_{j\in\mathcal{N}\backslash\{0\}}x^k_{0j} = 1  \quad &\forall&\  k \in \mathcal{K};\\
		& \sum_{k\in\mathcal{K}}\sum_{i\in\mathcal{N}\backslash\{0,j\}}x^k_{ij} = 1 \quad &\forall&\  j \in \mathcal{N};\\
		& \sum_{i\in\mathcal{N}\backslash\{j\}}x^k_{ij} = \sum_{h\in\mathcal{N}\backslash\{j\}}x^k_{jh} \quad &\forall&\  k \in \mathcal{K}, j \in \mathcal{N}\backslash\{0\};\\
		& \sum_{i\in\mathcal{N}}\sum_{j\in\mathcal{N}\backslash\{i\}}d_{ij}x^k_{ij} \le D_{max} \quad &\forall&\  k \in \mathcal{K};\\
		& \sum_{i\in\mathcal{N}}\sum_{j\in\mathcal{N}\backslash\{0,i\}}w_{j}x^k_{ij} \le W_{max} \quad &\forall&\  k \in \mathcal{K};\\
		& \sum_{i\in\mathcal{N}}\sum_{j\in\mathcal{N}\backslash\{0,i\}}v_{j}x^k_{ij} \le V_{max} \quad &\forall&\  k \in \mathcal{K};\\
		& M\cdot (1-x^k_{ij}) + y_{j} \ge y_i + t_{un} + t_{ij} \quad &\forall&\  k \in \mathcal{K}, i, j\in \mathcal{N};\\
		& y_i + t_{un} + t_{i0} \le T_{max} \quad &\forall&\  i\in \mathcal{N}\backslash\{0\}\\
		& x^k_{ij} \in \{0,1\} \quad &\forall&\  k\in \mathcal{K},i, j\in \mathcal{N};\\
		& y_i \ge 0 \quad &\forall&\  i\in \mathcal{N}.			
	\end{flalign}


\section{Algorithms and solver}\label{algo}
	
	\subsection{The Clarke and Wright algorithm}
		\paragraph{}This classical algorithm was first proposed in 1964 by Clarke and Wright to solve CVRPs in which the number of vehicles if free (!!cite!!). The method starts with vehicle routes containing the depot and one other customers. The total cost under this situation could be treated as the upper bound of our objective function. At each step, we merge two routes if we could reduce the total cost by doing so. The higher the saving, the higher the priority. The process was described as follows.

Step 1. Create 50 vehicle routes between each customer and the distribution center.

Step 2. Compute the savings $s_{ij}=c_{i1}+c_{1j}-c_{ij}$ for $i,j = 1,2,3,...,50$, and $i \neq j$.

Step 3. Order the savings in a non-increasing fashion.

step 4. Consider two vehicle routes containing arcs (i,1) and (1,j), respectively. If $s_{ij}>0$ and the new route is a feasible solution, merge these routes by introducing arc (i,j) and by deleting arcs(i,1) and (1,j).

Step 5. Repeat step 4 until no further improvement is possible. Stop. 	
	
		\subsection{The least distance algorithm}
			\paragraph{}The least distance algorithm is another simple but convenient way to obtain a vehicle routes. The advantage of this algorithm is it could reduce the computing time. But generally, we could not get the optimal solution by this way. The procedure was shown as follows.
			
Step 1. Start from an arbitrary customer, find the nearest customer.
			
Step 2. Find the nearest customer of the last customer.

Step 3. Repeat step 2 over all the 50 customers.

Step 4. Cut the routes by the constraints and get the satisfactory solution.
			
		\subsection{Cplex solver} 
		
		
%Result
\section{Simulation}\label{simu}
	\paragraph{Simulation}	

%Conclusion
\section{Conclusion}\label{conclution}
	
% Table generated by Excel2LaTeX from sheet 'Sheet1'
\begin{table}[htbp]
  \centering
  \caption{Result through the Clarke and Wright algorithm}
    \begin{tabular}{lccccccc}
    \hline
    Truck & 1     & 2     & 3     & 4     & 5     & 6     & 7 \\
    \hline
    node 1 & 0     & 0     & 0     & 0     & 0     & 0     & 0 \\
    node 2 & 616   & 276   & 247   & 406   & 443   & 851   & 17 \\
    node 3 & 645   & 550   & 221   & 184   & 631   & 972   & 943 \\
    node 4 & 423   & 255   & 182   & 299   & 215   & 109   & 359 \\
    node 5 & 689   & 484   & 604   & 901   & 888   & 331   & 51 \\
    node 6 & 400   & 582   & 283   & 303   & 185   & 137   & 0 \\
    node 7 & 947   & 660   & 101   & 0     & 121   & 322   & 0 \\
    node 8 & 763   & 662   & 606   & 0     & 833   & 559   & 0 \\
    node 9 & 810   & 0     & 360   & 0     & 0     & 140   & 0 \\
    node 10 & 0     & 0     & 977   & 0     & 0     & 601   & 0 \\
    node 11 & 0     & 0     & 842   & 0     & 0     & 0     & 0 \\
    node 12 & 0     & 0     & 0     & 0     & 0     & 0     & 0 \\
    \hline
    \end{tabular}%
  \label{tab:CW}%
\end{table}%


% Table generated by Excel2LaTeX from sheet 'Sheet2'
\begin{table}[htbp]
  \centering
  \caption{Result through the Least Distance algorithm}
    \begin{tabular}{lrrrrrrrrr}
    \hline
    Truck & 1     & 2     & 3     & 4     & 5     & 6     & 7     & 8     & 9 \\
    \hline
    node 1 & 0     & 0     & 0     & 0     & 0     & 0     & 0     & 0     & 0 \\
    node 2 & 221   & 331   & 359   & 559   & 631   & 645   & 810   & 888   & 943 \\
    node 3 & 182   & 322   & 51    & 140   & 443   & 616   & 276   & 121   & 360 \\
    node 4 & 604   & 137   & 185   & 601   & 215   & 0     & 763   & 901   & 606 \\
    node 5 & 283   & 972   & 833   & 17    & 303   & 0     & 947   & 299   & 101 \\
    node 6 & 0     & 109   & 851   & 423   & 582   & 0     & 0     & 406   & 400 \\
    node 7 & 0     & 184   & 0     & 0     & 660   & 0     & 0     & 842   & 689 \\
    node 8 & 0     & 977   & 0     & 0     & 662   & 0     & 0     & 484   & 0 \\
    node 9 & 0     & 0     & 0     & 0     & 247   & 0     & 0     & 255   & 0 \\
    node 10 & 0     & 0     & 0     & 0     & 0     & 0     & 0     & 550   & 0 \\
    \hline
    \end{tabular}%
  \label{tab:LeastDistance}%
\end{table}%


% Table generated by Excel2LaTeX from sheet 'Sheet3'
\begin{table}[htbp]
  \centering
  \caption{Comparison of two algorithms}
    \begin{tabular}{lcrcccl}
    \hline
          & \multicolumn{1}{l}{Truck} & \multicolumn{1}{l}{Distance} & \multicolumn{1}{l}{Fixed Cost} & \multicolumn{1}{l}{Travel Cost} & \multicolumn{1}{l}{Wait. Cost} & \multicolumn{1}{l}{Total Cost} \\
    \hline
    Least Dist.  & 7     & 1093.40 & 2100  & 15308 & 0     & 17408 \\
    C \& W & 9     & 1092.99 & 2700  & 15302 & 0     & 18002 \\
    \hline
    \end{tabular}%
  \label{tab:comparison}%
\end{table}%


	\paragraph{Result}
	The solution through the Clarke and Wright algorithm and the Least Distance algorithm were shown in Table \ref{tab:CW} and Table \ref{tab:LeastDistance}, respectively.  

	The comparison was displayed in Table \ref{tab:comparison}. Through the comparison, we see in this specific problem, Least Distance algorithm can obtain a better solution. Although C \& W algorithm could seek an less traveling distance, its routes need more truck, which caused the higher total cost.	

	
	\paragraph{Future research}On the one hand, algorithm is very crucial in vehicle routing problem. In our research, we simplify the model because we can not design an algorithm which can solve this problem in an acceptable time. Thus, seeking more effective algorithm could be our future research area. On the other hand, VRP could be extended by adding constraints, such as heterogeneous fleet and multiply trips, etc. Besides,  as electric vehicle get more and more popular because of its economic and environment-friendly advantages, studying when and where to get vehicle recharged are also challenge but meaningful problems.
	    
\newpage

% References
\nocite{*}
\printbibliography

\newpage
% Appendix
% \appendix

%\begin{comment}
% \section*{Appendix I: Testing Graphs}
	
%\end{comment}

\end{document}

%\begin{figure}[htbp]
%\centering \includegraphics[width=4in]{graphs/fig1}
%\caption{Roman Empire and its eight provinces}
%\end{figure}	

%Matrix
%\begin{equation*}
%	\begin{aligned}
%	A_{m\times m}=\left[
%		\begin{array}{ccc}
%			a_{11} & \cdots & a_{1m} \\
%			\vdots & \ddots & \vdots \\
%			a_{m1} & \cdots & a_{mm}\\
%		\end{array}
%		\right],
%	\end{aligned}
%	\end{equation*}





















