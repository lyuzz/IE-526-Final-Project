% Begin
\documentclass[12pt]{article}

% Packages
\usepackage[top=2cm, bottom=3cm, left = 1in, right = 1in]{geometry}
\usepackage[fleqn]{amsmath}
\usepackage{amssymb}
\usepackage{anysize}
\usepackage{graphicx}
\usepackage{subcaption}
\usepackage{float}
\usepackage{comment}
\usepackage{multirow}
\usepackage{xcolor}
\usepackage[
	backend=biber,
	maxcitenames=2,
	maxbibnames=10,
	citestyle=authoryear,
	bibstyle=authoryear,
	isbn=false,
	url=false,
	doi=false]{biblatex}
\renewbibmacro{in:}{}
\captionsetup[subfigure]{labelformat=empty}

% Bib file
\addbibresource{bibfile.bib}

% Paper style
\geometry{a4paper}

% Equation (section.equation)
\numberwithin{equation}{section}

% Title
\title{Report}
\author{Zefeng Lyu, Zeyu Liu, Rui Zhou}
\date{}

% Begin Document
\begin{document}

%MakeTitle
\maketitle

%Introduction
\section{Introduction}

	\paragraph{Introduction}

%Model
\section{Model}
	
	\paragraph{}We model this problem as a Vehicle Routing Problem with Time Windows and Capacity Constraints (VRPTWCC)\parencite{Schneider2016}.  Consider a set of customers and a distribution center $\mathcal {N}=\{0,1,2,..., N\}$, in which 0 represents the distribution center and $1,2,...N$ represent customer locations. Deliveries happen on a graph $G(V, E)$, where $V=\{i: i\in \mathcal{N}\}$ denotes customer locations and the distribution center, $E=\{(i,j):i,j\in \mathcal{N}\}$ denotes the road between location $i$ and $j$.  Each customer $i$ has a package of volume $v_i$ and weight $w_i$. A truck is chosen from a set of trucks $\mathcal{K}=\{1,2,...K\}$ to load packages of several customers and deliver with the route that minimizes the cost. Here we assume that the number of trucks is given in advance. This assumption is relaxed later in simulation optimization. There is a fixed cost $c_f$ associated with each truck out for delivery and a travel cost $c_t$ accumulates per hour for each truck. Every hour that a customer waits for the delivery, a waiting cost $c_w$ will be generated to model the impatience of that customer. Trucks are assumed to travel in a constant speed, thus travel time between location $i$ and $j$ is also constant and already given. We use $d_{ij}$ and $t_{ij}$ to denote distance and travel time between location $i$ and $j$. When trucks are loading packages, the total volume and weight of packages should not exceed the maximal volume and weight, $V_{max}$ and $W_{Max}$, respectively. Powered by electricity, trucks have a travel limit $D_{max}$ before recharging. We assume that delivery starts at 9:00 am and customers expect to receive packages before 9:00 pm, which forms a delivery window of $T_{max}=12$ hours.
	
	\paragraph{}Route of a truck is chosen by a binary variable $x_{ij}^k$. If truck $k$ chooses to travel from $i$ to $j$, $x_{ij}^k$ is 1; otherwise it is 0. For the time windows, another variable $y_i$ is defined to record the time that customer $i$ is served. Note that in the model we choose the start time at 0, so that $y_0$ will always be 0. Objective is to minimize the total cost of operation and of customer waiting, which consists of three parts: fixed cost for each trucks, travel cost of each truck measured by time units and customer waiting cost also measured by time units. The problem can be formulated as a Mixed Integer Program.
	\begin{flalign}
	\text{\emph{Minimize}} \quad & \sum_{k\in \mathcal{K}}\sum_{j\in \mathcal{N}}c_f x^k_{0j}
			+\sum_{k\in\mathcal{K}}\sum_{i\in \mathcal{N}}\sum_{j\in \mathcal{N}\backslash\{i\}}c_d d_{ij}x^k_{ij} + \sum_{i\in \mathcal{N}\backslash\{0\}}c_w (y_i - T_i) ,\\
	s.t. \quad  & \sum_{j\in\mathcal{N}\backslash\{0\}}x^k_{0j} = 1  \quad &\forall&\  k \in \mathcal{K};\\
		& \sum_{k\in\mathcal{K}}\sum_{i\in\mathcal{N}\backslash\{0,j\}}x^k_{ij} = 1 \quad &\forall&\  j \in \mathcal{N};\\
		& \sum_{i\in\mathcal{N}\backslash\{j\}}x^k_{ij} = \sum_{h\in\mathcal{N}\backslash\{j\}}x^k_{jh} \quad &\forall&\  k \in \mathcal{K}, j \in \mathcal{N}\backslash\{0\};\\
		& \sum_{i\in\mathcal{N}}\sum_{j\in\mathcal{N}\backslash\{i\}}d_{ij}x^k_{ij} \le D_{max} \quad &\forall&\  k \in \mathcal{K};\\
		& \sum_{i\in\mathcal{N}}\sum_{j\in\mathcal{N}\backslash\{0,i\}}w_{j}x^k_{ij} \le W_{max} \quad &\forall&\  k \in \mathcal{K};\\
		& \sum_{i\in\mathcal{N}}\sum_{j\in\mathcal{N}\backslash\{0,i\}}v_{j}x^k_{ij} \le V_{max} \quad &\forall&\  k \in \mathcal{K};\\
		& M\cdot (1-x^k_{ij}) + y_{j} \ge y_i + t_{un} + t_{ij} \quad &\forall&\  k \in \mathcal{K}, i, j\in \mathcal{N};\\
		& y_i + t_{un} + t_{i0} \le T_{max} \quad &\forall&\  i\in \mathcal{N}\backslash\{0\}\\
		& x^k_{ij} \in \{0,1\} \quad &\forall&\  k\in \mathcal{K},i, j\in \mathcal{N};\\
		& y_i \ge 0 \quad &\forall&\  i\in \mathcal{N}.			
	\end{flalign}

%Result
\section{Simulation}
	\paragraph{Simulation}	

%Conclusion
\section{Conclusion}
	
	\paragraph{Conclusion}
	
\newpage

% References
\nocite{*}
\printbibliography

\newpage
% Appendix
% \appendix

%\begin{comment}
% \section*{Appendix I: Testing Graphs}
	
%\end{comment}

\end{document}

%\begin{figure}[htbp]
%\centering \includegraphics[width=4in]{graphs/fig1}
%\caption{Roman Empire and its eight provinces}
%\end{figure}	

%Matrix
%\begin{equation*}
%	\begin{aligned}
%	A_{m\times m}=\left[
%		\begin{array}{ccc}
%			a_{11} & \cdots & a_{1m} \\
%			\vdots & \ddots & \vdots \\
%			a_{m1} & \cdots & a_{mm}\\
%		\end{array}
%		\right],
%	\end{aligned}
%	\end{equation*}





















